\documentclass[11.5pt, aspectratio=169]{beamer}
\input{preamble}
\usepackage{config/presento}
\usepackage{framed}
\usepackage{stmaryrd}
\usepackage{amsmath}
\usepackage{wasysym}
% Presento style file
\usepackage{float,lipsum}
\floatstyle{boxed}
\usepackage{mathpartir}
% custom command and packages
\input{config/custom-command}
\usepackage{verbatim}
% Information
\title{Cross-tier web programming for curated databases: A case study}
\author{Simon Fowler}
\institute{University of Edinburgh}
\date{19th February 2020}

\begin{document}

% Title page
\begin{frame}[plain]
\titlepage


\hfill
\vspace{-1em}
$
\renewcommand*{\arraystretch}{1.8}
\begin{array}{r}
   \includegraphics[height=0.7cm, keepaspectratio]{images/logos/inf_uoe.png} \\
\end{array}
$
\end{frame}


\begin{frame}{Curated Scientific Databases}
\end{frame}

\begin{frame}{GtoPdb: IUPHAR/BPS Guide to PHARMACOLOGY}
  \begin{minipage}{0.525\textwidth}
    \begin{fullpageitemize}
    \item Popular curated database detailing around 3000 pharmacological targets (e.g.,
      receptors) and 9700 ligands (e.g., drugs)
      \vspace{1em}
    \item Initially IUPHAR-DB, started in 2003
      \vspace{1em}
    \item Written in Java. Scale (as of late 2019):
      \begin{itemize}
        \itemR \emph{89MB} of data in \emph{181 tables}
        \itemR \emph{17935 lines} of data transformation code
        \itemR \emph{28819 lines} of JSP rendering code
        \itemR \emph{43129 lines} of data access code
      \end{itemize}
    \end{fullpageitemize}
  \end{minipage}
  \hfill
  \begin{minipage}{0.4\textwidth}
    \includegraphics[width=\textwidth]{images/gtopdb-screenshot.png}
  \end{minipage}

\end{frame}

\begin{frame}{The vision: Language support for curation}
\end{frame}

\begin{frame}{Cross-tier web programming}

  \begin{center}
    \includegraphics[width=0.75\textwidth]{images/3tier.pdf}
  \end{center}

  \begin{fullpageitemize}
  \item Curated databases normally a collection of applications:
    \begin{itemize}
      \itemR Database, web frontend, curation application
    \end{itemize}
    \vspace{1em}
  \item \textbf{Cross-tier} programming languages:
    \begin{itemize}
      \itemR Client, server, and database code written in same language
      \itemR \textbf{Links}: research cross-tier programming language developed
        at UoE since 2006
    \end{itemize}
    \vspace{1em}
  \item \emph{Goal: programming language support for curation functionality}
    \begin{itemize}
      \itemR Previous work has already added provenance tracking!
    \end{itemize}
  \end{fullpageitemize}
\end{frame}

\begin{frame}[fragile]{Language-Integrated Query}

  \begin{minipage}[t]{0.45\textwidth}
    {\large \textbf{SQL}}
    \begin{lstlisting}[language=sql]
SELECT name FROM ligand
WHERE approved
    \end{lstlisting}
  \end{minipage}
  %
  \hfill
  %
  \begin{minipage}[t]{0.45\textwidth}
    {\large \textbf{Links}}
    \begin{lstlisting}[language=Links]
for (l <-- ligand)
  where (l.approved)
  [ l.name ]
    \end{lstlisting}
  \end{minipage}

  \begin{fullpageitemize}
  \item \emph{Language-integrated Query}: Allow database queries to be written in host language, rather than SQL
  \item \emph{Benefits}:
    \begin{itemize}
      \itemR Full static typechecking of queries
      \itemR Only one language to learn
      \itemR Abstraction: parts of queries can be reused
    \end{itemize}
  \end{fullpageitemize}
\end{frame}

\begin{frame}[fragile]{Nested Queries}

  \begin{lstlisting}[language=Links]
for ( l <-- ligand )
  [( name = l.name,
     synonyms =
       for ( l2s <-- ligand2synonym )
         where ( l2s.ligand_id == l.ligand_id )
         [ l2s.synonym ]
  )]
  \end{lstlisting}

  \begin{fullpageitemize}
    \item Links queries need not be flat: can have arbitrarily-nested collections
    \item \emph{Extensively} used in GtoPdb case study
    \item \emph{Static bound on query count}: number of collection types appearing in output
  \end{fullpageitemize}

\end{frame}

\begin{frame}{Implementing GtoPdb in Links}
\end{frame}

\framecard{{\color{white}\bigtext{Evaluation}}}

\begin{frame}{Methodology}

  \begin{fullpageitemize}
  \item<1->{\Large \textbf{Query count} }
    \begin{itemize}
      \itemR Number of queries performed per page load
      \itemR \emph{Expectation}: Links lower due to static query bounds
    \end{itemize}
    \vspace{1em}
  \item<2->{\Large \textbf{Query handling time} }
    \begin{itemize}
      \itemR Time taken to execute query and process results into data
        structures
      \itemR \emph{Expectation}: Either Links performs better due to fewer queries, or
        Java performs better due to faster marshalling
    \end{itemize}
    \vspace{1em}
  \item<3->{\Large \textbf{Page build time} }
    \begin{itemize}
      \itemR Time between request handler invocation and response being ready
%     \itemR Useful to consider both including \emph{and} excluding query
%       handling time
      \itemR \emph{Expectation}: Java to perform better due to JIT and maturity
    \end{itemize}
  \end{fullpageitemize}
    \vspace{1em}

    \onslide<4->{
  Performance measurements performed locally, using an instrumented Links
  interpreter and modified GtoPdb code. Sample of 150 random pages from
  object data and disease data pages.
}
\end{frame}

\begin{frame}{Query Count}

  \begin{minipage}[t]{0.45\textwidth}
    \centering
    \includegraphics[scale=0.4]{images/objectdisplay_querycount_box.pdf}

    \begin{center}
      Object data page
    \end{center}
  \end{minipage}
  \hfill
  \begin{minipage}[t]{0.45\textwidth}
    \centering
    \includegraphics[scale=0.4]{images/diseasedisplay_querycount_box.pdf}

    \begin{center}
      Disease data page
    \end{center}
  \end{minipage}
  \vspace{1em}

  \begin{fullpageitemize}
  \itemR As expected, Links query count lower and more predictable
  \end{fullpageitemize}
\end{frame}

\begin{frame}{Query Handling Time}

  \begin{minipage}[t]{0.45\textwidth}
    \centering
    \includegraphics[scale=0.4]{images/objectdisplay_querytime_box.pdf}

    \begin{center}
      Object data page
    \end{center}
  \end{minipage}
  \hfill
  \begin{minipage}[t]{0.45\textwidth}
    \centering
    \includegraphics[scale=0.4]{images/diseasedisplay_querytime_box.pdf}

    \begin{center}
      Disease data page
    \end{center}
  \end{minipage}
  \vspace{1em}

  \begin{fullpageitemize}
  \itemR Links far more predictable due to more predictable query counts
  \itemR Median query time slightly higher on object display page, likely due to marshalling
  \end{fullpageitemize}
\end{frame}

\begin{frame}{Page Build Time}

\begin{minipage}[t]{0.45\textwidth}
    \centering
    \includegraphics[scale=0.3]{images/objectdisplay_pagebuild_excl_box.pdf}
    ~
    \includegraphics[scale=0.3]{images/objectdisplay_pagebuild_incl_box.pdf}

    \begin{center}
      Object data page
    \end{center}
  \end{minipage}
  \hfill
  \begin{minipage}[t]{0.45\textwidth}
    \centering
    \includegraphics[scale=0.3]{images/diseasedisplay_pagebuild_excl_box.pdf}
    ~
    \includegraphics[scale=0.3]{images/diseasedisplay_pagebuild_incl_box.pdf}

    \begin{center}
      Disease data page
    \end{center}
  \end{minipage}
  \vspace{1em}

  \begin{fullpageitemize}
  \itemR As expected, Java page generation \emph{much} faster due to maturity
  \itemR Outliers in Links due to (slow) parsing of data fields for large pages
  \end{fullpageitemize}
\end{frame}

\begin{frame}{Disease and Ligand Lists}

  \centering
  \includegraphics[scale=0.3]{images/diseaselist_stacked.pdf}
  \vspace{1em}

  \begin{fullpageitemize}
  \itemR List of all diseases and ligands
  \itemR Query counts in Java: 8995 (disease list) and 30479 (ligand list)
  \itemR \emph{Query counts in Links: 2!}
    \begin{itemize}
      \itemR \emph{Much} lower in Links due to nested queries \& no ORM used
    \end{itemize}
  \end{fullpageitemize}

\end{frame}


\framecard{{\color{white}\bigtext{Future Work}}}

\begin{frame}{Curation Interface}
\end{frame}

\begin{frame}{Temporal Data Management}
\end{frame}

\begin{frame}{Conclusion}
\end{frame}

\end{document}
