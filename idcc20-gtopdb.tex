\documentclass[11.5pt, aspectratio=169]{beamer}
\input{preamble}
\usepackage{config/presento}
\usepackage{framed}
\usepackage{stmaryrd}
\usepackage{amsmath}
\usepackage{wasysym}
% Presento style file
\usepackage{float,lipsum}
\floatstyle{boxed}
\usepackage{mathpartir}
% custom command and packages
\input{config/custom-command}
\usepackage{verbatim}
% Information
\title{Cross-tier web programming for curated databases: A case study}
\author{Simon Fowler$^1$, Simon D.\ Harding$^1$, Joanna Sharman$^2$, James Cheney$^1$}
\institute{University of Edinburgh$^1$\\
  Novo Nordisk$^2$}
\date{19th February 2020}

\begin{document}

% Title page
\begin{frame}[plain]
\titlepage


\hfill
\vspace{-1em}
$
\renewcommand*{\arraystretch}{1.8}
\begin{array}{r}
   \includegraphics[height=0.7cm, keepaspectratio]{images/logos/inf_uoe.png} \\
   \includegraphics[height=1.25cm,
   keepaspectratio]{images/logos/erc-logo.jpg} \\
\end{array}
$
\end{frame}


\begin{frame}{Curated Scientific Databases}

  \begin{fullpageitemize}
  \item {\LARGE \textbf{Curated Databases:}}
    \begin{itemize}
      \itemR ``databases that are populated and updated with a great deal of human effort'' (Buneman et al., 2008)
      \itemR Examples: CIA World Factbook, GtoPdb
    \end{itemize}
    \vspace{1em}
  \item {\LARGE \textbf{Curation challenges:}}
    \begin{itemize}
      \itemR Annotation propagation, provenance
        tracking, citation, archiving\ldots
      \itemR Have to be hand-crafted for every curated DB ---
        often with very stretched resources
    \end{itemize}
    \vspace{1em}
  \item {\LARGE \textbf{Ultimate goal: programming
      language support for curation
  functionality}}
    \begin{itemize}
      \itemR Previous work (Fehrenbach \& Cheney, 2018) has already added
        provenance tracking!
    \end{itemize}
  \end{fullpageitemize}
\end{frame}

\begin{frame}{GtoPdb: IUPHAR/BPS Guide to PHARMACOLOGY}
  \begin{minipage}{0.525\textwidth}
    \begin{fullpageitemize}
    \item Popular curated database detailing around 3000 pharmacological targets (e.g.,
      receptors) and 9700 ligands (e.g., drugs)
      \vspace{1em}
    \item Initially IUPHAR-DB, started in 2003
      \vspace{1em}
    \item Manually updated by 2-3 full-time curation staff at
      UoE, on advice of subcommittees worldwide
      \vspace{1em}
    \item Written in Java. Scale (as of late 2019):
      \begin{itemize}
        \itemR \emph{89MB} of data in \emph{181 tables}
        \itemR \emph{17935 lines} of data transformation code
        \itemR \emph{28819 lines} of JSP rendering code
        \itemR \emph{43129 lines} of data access code
      \end{itemize}
    \end{fullpageitemize}
  \end{minipage}
  \hfill
  \begin{minipage}{0.4\textwidth}
    \includegraphics[width=\textwidth]{images/gtopdb-screenshot.png}
  \end{minipage}
\end{frame}

\begin{frame}{Cross-tier web programming}

  \begin{center}
    \includegraphics[width=0.75\textwidth]{images/3tier.pdf}
  \end{center}

  \begin{fullpageitemize}
  \item Curated databases normally a
    \emph{collection of applications}:
    \begin{itemize}
      \itemR Database, web frontend, curation application
    \end{itemize}
    \vspace{1em}
  \item \textbf{Cross-tier} programming languages:
    \begin{itemize}
      \itemR Client, server, and database code written in same language
      \itemR \textbf{Links}: research cross-tier programming language developed
        at UoE since 2006
    \end{itemize}
    \vspace{1em}
  \item Before considering curation support, \emph{can cross-tier programming
    languages handle a real-world case study like GtoPdb?}
  \end{fullpageitemize}
\end{frame}


\framecard{{\color{white}\bigtext{Implementing GtoPdb in Links}}}

\begin{frame}[fragile]{GtoPdb Page Lifecycle}

  \begin{minipage}{0.3\textwidth}
     % TODO: Diagram (TiKZ?)
    \begin{tikzpicture}[node distance=1.5cm]
      \node (0) [diagnode] { Receive request };
      \node (1) [diagnode, below of=0]{ Query relevant data };
      \node (2) [diagnode, below of=1]{ Parse data fields };
      \node (3) [diagnode, below of=2]{ Query referenced ligands and references };
      \node (4) [diagnode, below of=3]{ Generate page };

      \draw [arrow] (0) -- (1);
      \draw [arrow] (1) -- (2);
      \draw [arrow] (2) -- (3);
      \draw [arrow] (3) -- (4);
    \end{tikzpicture}
  \end{minipage}
  \hfill
  \begin{minipage}{0.575\textwidth}
    {\small
      \begin{verbatim}
Some substituted benzazepines such as SKF-83959
are G-protein biased agonists of the dopamine
D<sub>1</sub> receptor and fail to activate
&beta;-arrestin recruitment <Reference id=28036/>;
their ability to signal through
G<sub>q</sub>-mediated pathways has been
controversial <Reference id=33435/>.
\end{verbatim}
  }

  {
    \centering
      \vspace{1em}
    \Huge
      \hfill $\Downarrow$ \hfill
      \vspace{1em}
  }

  {
    \small
Some substituted benzazepines such as SKF-83959
are G-protein biased agonists of the dopamine
D$_1$ receptor and fail to activate
$\beta$-arrestin recruitment [{\color{blue} 17}];
their ability to signal through
G$_q$-mediated pathways has been
controversial [{\color{blue} 32}].
  }
  \end{minipage}


\end{frame}

\begin{frame}[fragile]{Language-Integrated Query}

  \begin{minipage}[t]{0.45\textwidth}
    {\large \textbf{SQL}}
    \begin{lstlisting}[language=sql]
SELECT name FROM ligand
WHERE approved
    \end{lstlisting}
  \end{minipage}
  %
  \hfill
  %
  \begin{minipage}[t]{0.45\textwidth}
    {\large \textbf{Links}}
    \begin{lstlisting}[language=Links]
for (l <-- ligand)
  where (l.approved)
  [ l.name ]
    \end{lstlisting}
  \end{minipage}

  \begin{fullpageitemize}
  \item {\Large \emph{Language-integrated Query:}}
    \begin{itemize}
      \itemR Allow database queries to be written in host language, rather than SQL
    \end{itemize}
    \vspace{1em}
    \item { \Large \emph{Benefits}:}
    \begin{itemize}
      \itemR Full static typechecking of queries
      \itemR Only one language to learn
      \itemR Abstraction: parts of queries can be reused
    \end{itemize}
  \end{fullpageitemize}
\end{frame}

\begin{frame}[fragile]{Nested Queries}

  \begin{lstlisting}[language=Links]
for ( l <-- ligand )
  [( name = l.name,
     synonyms =
       for ( l2s <-- ligand2synonym )
         where ( l2s.ligand_id == l.ligand_id )
         [ l2s.synonym ]
  )]
  \end{lstlisting}

  \begin{fullpageitemize}
    \item Unlike in SQL, Links queries need not be flat: can have arbitrarily-nested collections
    \item \emph{Extensively} used in GtoPdb case study
    \item \emph{Static bound on query count}: number of collection types appearing in output
  \end{fullpageitemize}

\end{frame}

\begin{frame}{Status}

  \begin{minipage}{0.45\textwidth}
    \begin{center}
      \small
      \textbf{Official GtoPdb}
    \end{center}
    \centering
    \includegraphics[width=0.75\textwidth]{images/screen-gtopdb-orig.png}
  \end{minipage}
  \hfill
  \begin{minipage}{0.45\textwidth}
    \begin{center}
      \small
      \textbf{Links Reimplementation}
    \end{center}

    \centering
    \includegraphics[width=0.75\textwidth]{images/screen-gtopdb-links.png}
  \end{minipage}

  \vspace{1em}

  {\large
  \begin{fullpageitemize}
  \item All major data pages reimplemented
  \item 10981 lines of Links code, about 4 months of work
  \item \emph{Underlying database unchanged}
  \end{fullpageitemize}
  }
\end{frame}

\framecard{{\color{white}\bigtext{Evaluation}}}

\begin{frame}{Methodology}

  \begin{fullpageitemize}
  \item<1->{\Large \textbf{Query count} }
    \begin{itemize}
      \itemR Number of queries performed per page load
      \itemR \emph{Expectation}: Links lower due to static query bounds
    \end{itemize}
    \vspace{1em}
  \item<2->{\Large \textbf{Query handling time} }
    \begin{itemize}
      \itemR Time taken to execute query and process results into data
        structures
      \itemR \emph{Expectation}: Either Links performs better due to fewer queries, or
        Java performs better due to faster marshalling
    \end{itemize}
    \vspace{1em}
  \item<3->{\Large \textbf{Page build time} }
    \begin{itemize}
      \itemR Time between request handler invocation and response being ready
%     \itemR Useful to consider both including \emph{and} excluding query
%       handling time
      \itemR \emph{Expectation}: Java to perform better due to JIT and maturity
    \end{itemize}
  \end{fullpageitemize}
    \vspace{1em}

    \onslide<4->{
  Performance measurements performed locally, using an instrumented Links
  interpreter and modified GtoPdb code. Sample of 150 random pages from
  object data and disease data pages.
}
\end{frame}

\begin{frame}{Query Count}

  \begin{minipage}{0.28\textwidth}
    \centering
    \includegraphics[scale=0.4]{images/objectdisplay_querycount_box.pdf}

    \begin{center}
      Object data page
    \end{center}
  \end{minipage}
  \hfill
  \begin{minipage}{0.28\textwidth}
    \centering
    \includegraphics[scale=0.4]{images/diseasedisplay_querycount_box.pdf}

    \begin{center}
      Disease data page
    \end{center}
  \end{minipage}
  \hfill
  \begin{minipage}{0.375\textwidth}
    \raggedright
    As expected, Links query count \emph{lower} and \emph{more predictable}
  \end{minipage}
\end{frame}

\begin{frame}{Query Handling Time}

  \begin{minipage}{0.28\textwidth}
    \centering
    \includegraphics[scale=0.4]{images/objectdisplay_querytime_box.pdf}

    \begin{center}
      Object data page
    \end{center}
  \end{minipage}
  \hfill
  \begin{minipage}{0.28\textwidth}
    \centering
    \includegraphics[scale=0.4]{images/diseasedisplay_querytime_box.pdf}

    \begin{center}
      Disease data page
    \end{center}
  \end{minipage}
  \hfill
  \begin{minipage}{0.375\textwidth}
    \raggedright
    \begin{fullpageitemize}
      \item Links far more predictable due to more predictable query counts
        \vspace{1em}
      \item Median query time slightly higher on object display page, likely due to marshalling
    \end{fullpageitemize}
  \end{minipage}
\end{frame}

\begin{frame}{Page Build Time}

\begin{minipage}[t]{0.48\textwidth}
    \centering
    \includegraphics[scale=0.33]{images/objectdisplay_pagebuild_excl_box.pdf}
    ~
    \includegraphics[scale=0.33]{images/objectdisplay_pagebuild_incl_box.pdf}

    \vspace{-0.75em}
    \begin{center}
      Object data page
    \end{center}
  \end{minipage}
  \hfill
  \begin{minipage}[t]{0.48\textwidth}
    \centering
    \includegraphics[scale=0.33]{images/diseasedisplay_pagebuild_excl_box.pdf}
    ~
    \includegraphics[scale=0.33]{images/diseasedisplay_pagebuild_incl_box.pdf}

    \vspace{-0.75em}
    \begin{center}
      Disease data page
    \end{center}
  \end{minipage}
  \vspace{1em}

  \begin{fullpageitemize}
  \itemR As expected, Java page generation \emph{much} faster due to maturity
  \itemR Outliers in Links due to (slow) parsing of data fields for large pages
  \end{fullpageitemize}
\end{frame}

\begin{frame}{Disease and Ligand Lists}

  \begin{minipage}{0.375\textwidth}
  \centering
  \includegraphics[height=0.8\textheight]{images/diseaselist_stacked.pdf}
  \end{minipage}
\hfill
  \begin{minipage}{0.6\textwidth}
  \begin{fullpageitemize}
  \item {\large List of all diseases and ligands}
    \vspace{1em}
  \item {\large Query counts in Java}
    \begin{itemize}
      \itemR 8995 (disease list)
      \itemR 30479 (ligand list)
    \end{itemize}
    \vspace{1em}
  \item {\large \emph{Query counts in Links: 2!}}
    \begin{itemize}
      \itemR \emph{Much} lower due nested queries \& only relevant data queried
    \end{itemize}
  \end{fullpageitemize}
  \end{minipage}

\end{frame}


\framecard{{\color{white}\bigtext{Wrapping Up}}}

\begin{frame}{Ongoing Work: Curation Interface}

  \begin{minipage}{0.45\textwidth}
  \begin{center}
    \includegraphics[width=\textwidth]{images/curation-interface.png}
  \end{center}
  \end{minipage}
  \hfill
  \begin{minipage}{0.5\textwidth}
  \begin{fullpageitemize}
  \item {\large \textbf{Ongoing work: Curation interface}}
    \begin{itemize}
      \itemR Implemented two illustrative pages from curation interface
      \itemR Aim to use these as realistic case study for curation functionality
    \end{itemize}
  \end{fullpageitemize}
\end{minipage}
\end{frame}

\begin{frame}[fragile]{Conclusion}

  \begin{fullpageitemize}
  \item {\LARGE \textbf{Summary}}
    \begin{itemize}
      \itemR \textbf{Curated databases}:
        \begin{itemize}
          \itemR Databases maintained using much human effort
          \itemR Consist of multiple applications: database, web
            frontend, curation interface
        \end{itemize}
%
      \itemR Cross-tier programming languages: client, server, database
        code in \emph{single language}
      \itemR First case study of reimplementing a large-scale curated database
        in a cross-tier programming language
      \itemR Performance results: Lower query counts; more predictable and often
        lower query times; but slower page build times
    \end{itemize}
  \end{fullpageitemize}

  {\large
  \begin{center}
    \verb+@Simon_JF+\\
    \verb+simon.fowler@ed.ac.uk+\\
    \url{http://www.links-lang.org}
  \end{center}
}
\end{frame}

\end{document}
